To calculate the mean and variance for the given probability density function (pdf), we'll follow the standard procedures using integrals.

### 1. *Mean (Expected Value)*

The mean (expected value) of a continuous random variable \( X \) with pdf \( p_X(x) \) is given by:

\[
\mu = \mathbb{E}[X] = \int_{-\infty}^{\infty} x \, p_X(x) \, dx
\]

For the uniform distribution \( p_X(x) \) defined on the interval \([a, b]\), the pdf is:

\[
p_X(x) = \frac{1}{b - a}, \quad \text{for } a \leq x \leq b
\]

Thus, the mean becomes:

\[
\mu = \int_{a}^{b} x \cdot \frac{1}{b - a} \, dx
\]

To solve this:

\[
\mu = \frac{1}{b - a} \int_{a}^{b} x \, dx
\]

Now, integrate \( x \):

\[
\int x \, dx = \frac{x^2}{2}
\]

So,

\[
\mu = \frac{1}{b - a} \left[ \frac{x^2}{2} \right]_a^b = \frac{1}{b - a} \left( \frac{b^2}{2} - \frac{a^2}{2} \right)
\]

Simplifying:

\[
\mu = \frac{b^2 - a^2}{2(b - a)}
\]

This can be factored as:

\[
\mu = \frac{(b - a)(b + a)}{2(b - a)} = \frac{a + b}{2}
\]

So, the *mean* is:

\[
\mu = \frac{a + b}{2}
\]

---

### 2. *Variance*

The variance \( \sigma^2 \) of a continuous random variable \( X \) is defined as:

\[
\sigma^2 = \mathbb{E}[X^2] - (\mathbb{E}[X])^2
\]

We already know that \( \mathbb{E}[X] = \mu = \frac{a + b}{2} \).

Next, we calculate \( \mathbb{E}[X^2] \):

\[
\mathbb{E}[X^2] = \int_{a}^{b} x^2 \cdot \frac{1}{b - a} \, dx
\]

To solve this:

\[
\mathbb{E}[X^2] = \frac{1}{b - a} \int_{a}^{b} x^2 \, dx
\]

Now, integrate \( x^2 \):

\[
\int x^2 \, dx = \frac{x^3}{3}
\]

So,

\[
\mathbb{E}[X^2] = \frac{1}{b - a} \left[ \frac{x^3}{3} \right]_a^b = \frac{1}{b - a} \left( \frac{b^3}{3} - \frac{a^3}{3} \right)
\]

Simplifying:

\[
\mathbb{E}[X^2] = \frac{b^3 - a^3}{3(b - a)}
\]

We can factor the numerator:

\[
b^3 - a^3 = (b - a)(b^2 + ab + a^2)
\]

Thus,

\[
\mathbb{E}[X^2] = \frac{(b - a)(b^2 + ab + a^2)}{3(b - a)} = \frac{b^2 + ab + a^2}{3}
\]

Now, the *variance* \( \sigma^2 \) is:

\[
\sigma^2 = \mathbb{E}[X^2] - \mu^2
\]

Substituting \( \mathbb{E}[X^2] \) and \( \mu \):

\[
\sigma^2 = \frac{b^2 + ab + a^2}{3} - \left( \frac{a + b}{2} \right)^2
\]

Simplify \( \left( \frac{a + b}{2} \right)^2 \):

\[
\left( \frac{a + b}{2} \right)^2 = \frac{a^2 + 2ab + b^2}{4}
\]

Now, subtract:

\[
\sigma^2 = \frac{b^2 + ab + a^2}{3} - \frac{a^2 + 2ab + b^2}{4}
\]

To subtract, we need a common denominator. The common denominator between 3 and 4 is 12, so we rewrite the terms:

\[
\sigma^2 = \frac{4(b^2 + ab + a^2)}{12} - \frac{3(a^2 + 2ab + b^2)}{12}
\]

Now, perform the subtraction:

\[
\sigma^2 = \frac{4(b^2 + ab + a^2) - 3(a^2 + 2ab + b^2)}{12}
\]

Simplifying the numerator:

\[
4(b^2 + ab + a^2) - 3(a^2 + 2ab + b^2) = b^2 + a^2 - 2ab
\]

So,

\[
\sigma^2 = \frac{b^2 - 2ab + a^2}{12} = \frac{(b - a)^2}{12}
\]

Thus, the *variance* is:

\[
\sigma^2 = \frac{(b - a)^2}{12}
\]

---

### Summary:
- *Mean* \( \mu = \frac{a + b}{2} \)
- *Variance* \( \sigma^2 = \frac{(b - a)^2}{12} \)

These are the mean and variance for the given uniform distribution.
